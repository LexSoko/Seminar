%---------------DOCUMENT SETTINGS------------------------------------
%\documentclass[headheight=30pt]{scrartcl}
\documentclass[12pt]{scrartcl}

\usepackage[utf8]{inputenc} % utf8x durch utf8 ersetzt wegen biblatex
\usepackage[T1]{fontenc}
\usepackage{amsmath,amssymb,amstext,amsfonts}
\usepackage[ngerman]{babel}
\usepackage{csquotes}
\usepackage{pdfpages} %zum importieren von pdf dateien
\usepackage{geometry}
\usepackage[headsepline]{scrlayer-scrpage}
\usepackage{lastpage}
%\usepackage[style=numeric]{biblatex} %Literaturverwaltung
\usepackage{float}
\usepackage{textcomp}
\usepackage{gensymb}
\usepackage{physics}
\usepackage{graphicx}
\usepackage[export]{adjustbox}
\usepackage{a4wide}
\usepackage{siunitx}
\usepackage{hyperref}
\usepackage{multicol}
\usepackage{makecell}
%für seminararbeit
\usepackage{doi}
\usepackage[compress]{cite}

\geometry{      %holt mehr aus einer A4 Seite raus
    a4paper,
    total={170mm,257mm},
    left=20mm,
    top=25mm,
   }
\graphicspath{ {./nudes/} }  %pfad für bilder


\hypersetup{colorlinks=false}

%\addbibresource{literatur.bib} %Literatur-Resourcen
%\setlength{\headheight}{0.0pt} % macht zwar headheight warnings aber dafür nutz latex die seitengröße besser aus.
\pagestyle{scrheadings}
\newcommand{\subf}[2]{
    {
        \begin{tabular}[c]{@{}c@{}}
            {\setlength{\extrarowheight}{100pt} #1 }\\#2
        \end{tabular}
    }
}
\newcommand{\allc}{\multicolumn{1}{c|}{-}}
\newcommand{\monofig}[4]{
    
    \begin{figure}[H]
    \centering
    \includegraphics[#1]{#2}
    \caption{
        #3
    }
    \label{#4}
    \end{figure}
    
}
\newcommand{\polyfig}[6]{
    \begin{figure}[H]
        \centering
        \begin{minipage}[b]{0.45\textwidth}
            \centering
            \includegraphics[width=0.9\textwidth]{#1}
            \caption{#2}
            \label{#3}
        \end{minipage}
        \begin{minipage}[b]{0.45\textwidth}
            \centering
            \includegraphics[width=0.9\textwidth]{#4}
            \caption{#5}
            \label{#6}
        \end{minipage}
    \end{figure}
}
\newcommand{\code}[1]{
    \texttt{#1}
}

\date{\today{}, Location}
\author{Wilhelm Ecker , Aleksey Sokolov}
\title{Dokumentitel}

%---------------HEADER TEXT------------------------------------
\clearpairofpagestyles
\ihead{\today{} \\ }
\chead{Wilhelm Ecker / Aleksey Sokolov\\Elektronen Beschleuniger in der Medizin }
\ohead{Seminar\\ }
\cfoot{\pagemark \, / \, \pageref{LastPage}}

%---------------DOCUMENT TEXT------------------------------------
\begin{document}
%\includepdf[]{Deckblatt.pdf} %Insert title page NaWi-Graz
\tableofcontents
\newpage

SEAS TU
\cite{ThontadaryaUmakantha:1971}
dsgadggüs
\section{Abstract}
\section{Diagnostik}
%willy 
%https://www.bfs.de/DE/themen/ion/anwendung-medizin/diagnostik/roentgen/roentgen-verfahren.html
\subsection{Röntgenaufnahme}
%typischer wellenlängenbereich für abbildung (was für ein material wird zur erzeugung der röntgenstrahlung benutzt)


%grund für verschiedenen Kontrast (material spezifisch)
%vergleich alte technik neue (geschichtlich)
%arten von detektoren (film folien & oder photodioden)

%typische anwendung
%gefahren 
%strahlenbelastung

\subsection{Röntgendurchleuchtung (Fluoroskopie)}
%typischer wellenlängenbereich für abbildung (was für ein material wird zur erzeugung der röntgenstrahlung benutzt)

%geschichtliche analyse
%Anwendung & technische umsetzung
%Angiographie (kontrastmittel verabreicht)
%Arten von detektoren

%gefahren
%strahlenbelastung

\subsection{Computertomographie}
%technische Umsetzung der Bildgebung und erzeugung von Röntgenstrahlen (wellenlängenbereich)
%Detektion
%schnittbildererzeugung 
%geschichtliche analyse


%gefahren
%strahlenbelastung & Low-Dose-CT

%Andere tomographische Verfahren 
%neueste entwicklung in der medizintechnik
\section{X-Ray Therapie}
\bibliography{literatur.bib}
\bibliographystyle{seminarstyle}
\end{document}