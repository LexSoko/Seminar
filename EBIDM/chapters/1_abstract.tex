\section{Abstract}

Seit der Entdeckung der Röntgenstrahlung zählt diese zu den wichtigsten Methoden innerhalb technischer und medizinischer Anwendungen.
Von Strukturanalysen in der Materialwissenschaft bis hin zu hochauflösenden Bildern und Animationen in der Röntgendiagnostik haben 
sie die Wissenschaft und die Industrie stark geprägt. In dieser Arbeit wird sich mit mehreren Methoden der Röntgendiagnostik und 
Therapie in der Medizin beschäftigt. Insbesondere wird der Fokus auf den Beginn der 
Diagnostik mittels Folientechnik und dessen Entwicklung bis hin zu modernster Technik im Rahmen der Computertomographie gelegt, sowie
deren Anwendungsmöglichkeiten in der Therapie von Erkrankungen. Desweiteren wird der prinzipielle Aufbau eines herkömmlichen medizinischen Elektronenlinearbeschleuniger besprochen.
Dieser wurde im letzten Jahrhundert enorm weiterentwickelt, was neue, präzisere Behandlungsmethoden schafft.