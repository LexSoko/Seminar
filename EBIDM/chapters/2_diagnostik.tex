\section{Diagnostik}
\label{sec:Diagnostik}
%willy 
%https://www.bfs.de/DE/themen/ion/anwendung-medizin/diagnostik/roentgen/roentgen-verfahren.html
Röngtenaufnahmen sind seit der Entdeckung 1895 ein fixer Bestandteil der Medizin. Vorallem Fortschritte in der Computertechnologie
ermöglichen es, immer bessere bildgebende Verfahren zu entwickeln, die die Breite der radiologischen Untersuchungen enorm vergrößert
haben. Mit bis zu 80 \% stellen großflächige radiografische Aufgnahmen von Organen Skelett und Lunge den Hauptanteil in der radiologischen
Diagnostik dar \cite{MedizinischePhysik}.

\subsection{Röntgenaufnahme}
\label{sec:Röntgenaufnahme}
Bis heute stellt die Film- und Folientechnik eine weit genutzte Technik in der Diagnostik dar. Der Vorteil liegt hierbei
bei der einfachen Technik für die Aufnahme, der hohen Bildqualität, sowie im preiswerten Verhältnis von Nutzen zu den Kosten. Allerdings 
wird diese Technik in Industrieländern durch digitale Detektoren ersetzt, aufgrund der schlechen Möglichkeit für die Nachbearbeitung, 
der aufwendigen chemischen Entwicklung, sowie des hohen Dosisbedarfs an Strahlung \cite{MedizinischePhysik}.



\subsubsection{Film-Folientechnik}
\label{subsubsec:filmfolientechnik}
Ausgang ist hierbei ein Film oder ein Fluoreszenschirm als Empfänger für das Bild. Trifft Röntgenstrahlung auf den Film,
färbt sich dieser schwarz, je nach Dosis der Strahlung. Tifft die Strahlung auf Knochen oder Gewebe, wird diese absorbiert und 
und erscheint im Bild weiß oder gräulich. Da die Bildsignal direkt proportional zur Strahlungsdosis ist, so muss eine hohe Dosis 
aufgebracht werden, um gute Bilder zu erzeugen. Dieses System ist optimiert worden, indem man für den Film Verstärkerfolien entwickelt hat,
die die benötigte Dosis stark verringert haben \cite{Medizintechnik}. 

\subsubsection{Speicherfolien}
\label{subsubsec:Speicherfolien}
Bei dieser Technologie werden anstatt von Röngtenfilmen, Speicherfolien belichtet \cite{MedizinischePhysik}. Die Strahlung hebt die Elektronen in der
Kristallstruktur in ein höheres Energieniveau, wo diese für mehrere Stunden verweilen können. Um die Information auszulesen, wird 
die Folie aus der Kasette entfernt und mittels eines Laserstrahls abgetastet. Bei diesem Vorgang kehren die Elektronen unter
Abgabe von Fluoreszenlicht in ihren Ausgangszustand zurück. Dieses Licht wird von Photodetektoren detektiert, welche
im nächsten Schritt dieses in elektrische Signale umwandeln und digitalisieren. Nachdem der Vorgang beendet ist, kann die Folie 
mithilfe von sichtbarem Licht in den Ausgangszustand zurückversetzt und in die Kasette zurückgegeben werden \cite{Medizintechnik}.


\subsubsection{Flachdetektoren}
\label{subsubsec:Flachdetektoren}

Bei Flachdetektoren mit direkter Wandlung kommt es aufgrund einer Selenschicht zu sofortigen Umwandlung von Röntgenquanten in
elektrische Ladung. Trifft ein Röntgenquant auf die Schicht, wird eine Ladung erzeugt, die in einem Kondensator gespeichert wird. Mithilfe
eines Analog-Digital-Wandlers wird die Ladung weitergeleitet und kann zeilenweise ausgelesen werden. Somit kann ein digitales Bild
erstellt werden \cite{Artikel}. 

Bei den Detektoren mit indirekter Wandlung liegt eine Szintillatorschicht aus Cäsiumjodid auf Siliziumphotodioden, die als Matrix
angeordnet sind. In der Szintillatorschicht werden die Röntgenquanten in Lichtquanten umgewandelt, welche auf Detektorelemente umgeleitet 
werden. Die Ladung wird aus den Photdioden ausgelesen und mithilfe eines Analog-Digital-Wandlers in ein digitales Bild umgewandelt
 \cite{Artikel}.


%typischer wellenlängenbereich für abbildung (was für ein material wird zur erzeugung der röntgenstrahlung benutzt)


%grund für verschiedenen Kontrast (material spezifisch)
%vergleich alte technik neue (geschichtlich)
%arten von detektoren (film folien & oder photodioden)

%typische anwendung
%gefahren 
%strahlenbelastung

\subsection{Röntgendurchleuchtung (Fluoroskopie)}
\label{sec:Fluoroskopie}
Im Unterschied zu Röntgenaufnahmen kommt es bei der Fluoroskopie zu einer kontinuierlichen Strahlungsaussetzung mit Röntgenstrahlung,
 welche ein durchlaufendes Bild am Monitor liefern. Um ansonsten unsichtbare Organe am Bild sichtar zu machen, so werden Kontrastmittel eingesetzt.
Iodbasierte Materialien werden für Aterien, sowie die Herzkammern eingesetzt, bariumbasierte Mittel finden Anwendung im Magen-, Darmtrakt
\cite{Artikel2}.




\subsubsection{Dynamische Festkörperdetektoren}
\label{subsubsec:Festkörperdetektoren}
Diese nutzen ebenso Cäsiumjodid als Szintillator und erlauben den Betrieb nahe bei starken Magnetfeldern, zum Beispiel
wie bei Kernspintomographen. Sie sind geeignet für die Aufnahme von 3D-Bildern, aufgrund ihrer verzerrungsfreien und gleichmäßigen
Bilddarstellung. Angewendung finden sie in der Zahn- und HNO-Medizin, sowie bei der Durchleuchtung und Angiographie 
\cite{MedizinischePhysik}.

%typischer wellenlängenbereich für abbildung (was für ein material wird zur erzeugung der röntgenstrahlung benutzt)

%geschichtliche analyse
%Anwendung & technische umsetzung
%Angiographie (kontrastmittel verabreicht)
%Arten von detektoren

%gefahren
%strahlenbelastung
\subsection{Strahlenbelastung für Röntgenaufnahmen und Fluoroskopien}
\label{subsec:Strahlenbelastung}

Für die Untersuchungen mittels Röntgenaufnahmen und Fluoroskopien gelten je nach Anwendungsbereich am Körper unterschiedliche 
Richtwerte. Die Strahlungsdosis wird in der SI-Einheit 'Gray', abgekürzt 'Gy' angegeben. Für Messungen am Skelett gilt eine maximale
Empfängerdosis von $\leq$10 $\mu$Gy/Bild. Für den Rumpf und Kopf gelten $\leq$5 $\mu$Gy, sowie für die digitale Durchleuchtung $\leq$0.6 
 $\mu$Gy/s. Bei digitalen mammographischen Untersuchungen liegen die Grenzwerte bei $\leq$75 $\mu$Gy und $\leq$100 $\mu$Gy \cite{Artikel}.




\subsection{Computertomographie (CT)}
\label{sec:CT}
Die Computertomographie ist neben der klassischen Röntgendiagnostik eine der bedeutensten Methoden zur Bildgebung in der 
Diagnostik von Krankheiten \cite{Artikel3,Artikel4}. Bei dieser Technik wird der Patient mittels eines Röntgenstrahls 
abgetastet, welcher kreisförmig um ihn herumfährt. Mithilfe von optischen Effekten wird der Strahl so geformt, dass dieser flach 
aufgetragen wird und die Schichtdicke angibt. Ein Teil der Röntgenstrahlung wird beim Durchdringen des Körpers absorbiert und kann 
von den Detektoren bestimmt werden. Die an den Detektoren aufgenommene Strahlung kann im Anschluss zu Graustufen umgerechnet und 
bildlich dargestellt werden \cite{Artikel3}. 
Heutzutage können Schichtdicken von bis zu 0.24 mm gemessen werden, die gemittelte Strahlendosis beträgt hierbei 2-3 mSv am Körperstamm.
Verwendung findet die CT bei der Bestimmung von Frakturen bei Gelenken, des Beckens oder der Wirbelsäule. Ebenso findet man sie 
in der Orthopädik, zum Beispiel zur Beurteilung von postoperativen Zuständen und auch in der Tumordiagnostik \cite{Artikel4}.



%technische Umsetzung der Bildgebung und erzeugung von Röntgenstrahlen (wellenlängenbereich)
%Detektion
%schnittbildererzeugung 
%geschichtliche analyse


%gefahren
%strahlenbelastung & Low-Dose-CT

%Andere tomographische Verfahren 
%neueste entwicklung in der medizintechnik