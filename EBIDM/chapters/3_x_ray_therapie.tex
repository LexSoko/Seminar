
\section{X-Ray Therapie}
%aleksey
%https://www.radiologyinfo.org/en/info/linac

%https://altairusa.com/role-of-the-linear-accelerator-linac-in-cancer-radiation-therapy/
\subsection{Allgemein}

%geschichtlicher überblick
%behandelte krankheiten
%typische energieskalen

\subsubsection{Röntgentherapie}
Die Röntgentherapie hat sich im letzten Jahrhundert als eine der wichtigsten medizinischen Instrumente etabliert.
Vorallem im Bereich der Krebsbehandlung bietet sie eine gezielte und operationfreie Methode, um lokal Zellenstrukturen zu verändern.
Im Laufe einer Krebsbehandlung werden über 50\% der Patienten mit einer Form der Röntgentherapie behandelt.
Meist wird dies in Kombination mit einer präzisen Diagnostikmethode, wie der in Kapitel \ref{sec:Diagnostik} behandelten Computertomographie (CT), sowie
der Magnetresonanztomographie (MRT) oder Positron Emissions Tomographie (PET) vollzogen. \cite{hoskin2019external}
\subsubsection{Strahlungsquellen}
In der Röntgentherapie wird mit elektromagnetischer Strahlung im Röntgenbereich (>100 eV) gearbeitet. 
Diese ist ionisierend und kann Zellstrukturen auf der Molekularebene zerstören.
Jedoch können die Geräte, die die Strahlung mittels dem Beschleunigen von Elektronen erzeugen, diese auch als Projektil nutzen.
Hierbei spricht man von Elektronenstrahlung, welche zwischen 2 und 30 MeV liegt.
Neue Simulationen zeigen allerdings auch, dass Elektronenstrahlung bis zu 100 MeV eingesetzt werden kann, um tiefschichtigere Tumore gezielt angreifen zu können, ohne umliegende, gesunde Zellenstrukturen zu zerstören.
Dies kommt jedoch mit technischen Schwierigkeiten, da solch hohe Energien nur mit 1-5 m langen Beschleunigungsröhren realisierbar sind. \cite{shady_author}
Die häufigsten Strahlungsquellen in der Medizin sind Linearbeschleuniger.
Mit diesen kann die Energie der erzeugten Strahlung präzise eingestellt und leichter skaliert werden.
Andere Strahlungsquellen, wie Röntgenröhren, sind aufgrund der Limitierung der Röntgenspannung durch technische Faktoren wie der Dimensionierung, begrenzt einsetzbar.
Für die meisten Anwendungen in der Bestrahlungstherapie liegt die Energie der erzeugten Photonen bei 4-6 MeV, jedoch kann diese bei tiefschichtiger Bestrahlung auch bis 25 MeV reichen. \cite{KriegerHannoSfTu}









%Diese können Elektronen schon auf einer kurzen Distanz auf sehr hohe Geschwingkeiten bringen, was in einer hohen 






\subsection{Funktionsweise der verwendeten Geräte}
In der Regel werden in der Medizin Hochfrequenz-Beschleuniger verwendet, bei denen das interne elektrische Feld 
im Beschleunigungsrohr mit 3 GHz oszilliert. 
Moderne Elektronenlinearbeschleuniger werden digital gesteuert, sowie in Echtzeit überwacht und können durch ihre relativ kompakte Bauweise
mit Elektromotoren gedreht, sowie räumlich ausgerichtet werden.
Der Aufbau kann grundsätzlich in sechs Einheiten unterteilt werden: Energieversorgung, Modulator, Beschleunigungseinheit, Strahlenkopf, Bedienungseinheit und Verifikationsystem. \cite{KriegerHannoSfTu}

\monofig{width=0.6\textwidth}{Krieger_med_linAC_Aufbau.JPG}{Schematischer Aufbau eines medizinischen Elektronenbeschleunigers.
Mo: Modulator,
E: Energieversorgung,
HF:Hochfrequenztransportsystem,
K:Elektronenkanone,
B:Beschleunigungsrohr,
Ma: Umlenkmagnetsystem,
S: Strahlerkopf,
Iso: Drehachse (Isozentrumsachse),
G: Gantry (Beschleunigerarm), Bildquelle: Krieger H. \cite{KriegerHannoSfTu}}{fig:gestohlen}

Die Quelle der Hochfrequenzerzeugung, der Modulator, enthält die gesamte Steuerelektronik und ist direkt mit der Energieversorgung und dem Beschleunigungsrohr gekoppelt.
Diese Einheit kann bei kleineren Systemen direkt auf der Anlage verbaut, oder räumlich getrennt sein.
Die sogenannte Gantry (siehe Abbildung \ref{fig:gestohlen}) ist eine mit Präzision drehbare Einheit, in der sich das Vakuumsystem für die Strahlführung und das Beschleunigungsrohr befindet.
Dabei wird die verbaute Technik von Kühlaggregaten gekühlt.
Es gibt zwei Ausführungen der Beschleunigungeinheit, die grundsätzlich für die medizinische Behandlung keine Rolle spielen.
Man spricht von Wanderwellenbeschleunigern, die die Phasefokusierung nutzen, um Elektronenbündel mit einer positiven elektromagnetischen Wellenfront zu beschleunigen,
und Stehwellenbeschleunigern, die die Umpolung einer stehenden elektromagnetischen Welle und feldfreie Räume nutzen, um eine Beschleunigungswirkung zu erzielen.
Die Elektronen werden nach dem Beschleunigungsrohr im Strahlenkopf durch ein magnetisches Umlenksystem, welches zugleich für die Energieanalyse genutzt wird, auf ein Photonentarget fokussiert.
Nach der Emission der hochenergetischen Strahlung wird diese durch Blendensysteme, den Kollimatoren, auf ein einstellbares Feld projeziert.
Da die Strahlung isotrop ausgesendet wird, muss der gesamte Strahlenkopf abgeschirmt sein. Dies führt zu einer erheblichen Steigerung der Gantry Masse.
Trotzdem ist es möglich, die gesamte Gantry mit einer hohen Genauigkeit um die Isozentrumachse zu drehen.
Im Verifkationsystem werden weitgehend selbständig, unter Beobachtung der protokollierten Bestrahlungen, die Bestrahlungsparameter festgelegt.\cite{KriegerHannoSfTu}

\monofig{width=0.6\textwidth}{Linearbeschleuniger_Kiel.JPG}{Moderner medizinischer Linearbeschleuniger Bildquelle: Wikimedia-Commons (CC BY-SA 3.0) \cite{linacbild}}{fig:gestohlen_2}


%LinAC grobe funktionsweise
%adaptierung in der medizin
%häufig verwendete geräte im europäischen raum

\subsection{Anwendung und Arten von Behandlungsmethoden}
In den letzten 50 Jahren gab es große Fortschritte in der Behandlung von Tumoren mit hochenergetischer Strahlung.
Es gibt eine große Anzahl an verschiedenen Behandlungsmethoden, welche teils nur Modifizierungen von bewährten Methoden sind.
Da alle zu behandeln den Rahmen dieser Seminararbeit sprengen würde, wurden die in der Medizin am weitesten verbreiteten behandelt.
\subsubsection{Intensity-Modulated Radiation Therapy (IMRT)}
IMRT hat sich als eine der wichtigsten Behandlungsmethoden behauptet, da sie sehr präzise, lokale Strahlungsdosen verteilt, ohne gesundes Gewebe beträchtlich zu beeinträchtigen.
Geschafft wird dies mit Computeralgorithmen, welche die optimale Dosis, sowohl als auch den optimalen Bestrahlungsbereich bestimmen können.
Dabei kann man auch sehr akkurat die Strahlenbelastung an Grenzflächen zu gesundem Gewebe mit Multilamellenkollimatoren einstellen.
Das Funktionsprinzip der IMRT ähnelt der Diagnostikmethode CT (Computertomographie), bei der die Gantry um den Patienten rotiert und Schicht für Schicht eine materialkontrastreiche Abbildung des Gewebes erstellt.
Jedoch wird hierbei nicht die ganze zu untersuchende Körperstelle bestrahlt, sondern nur die Tumorzellen.
Der Computer bestimmt eigenständig nach dem Analysieren der ersten Bestrahlung, welche inhomogene Strahlungsverteilung gewählt werden muss, um eine lokal eingeschränkte, homogene Dosisverteilung im Tumor zu erhalten. \cite{teh1999intensity}
Der Vorteil dieser Therapie ist, dass Strahlung in einzelne, von einander getrennte Subbereiche mit verschiedener Intensität sowie Einstrahlwinkel aufgeteilt werden kann.
\subsubsection{Volumetric Modulated Arc Therapy (VMAT)}
Diese Art der Therapie wurde 2007 vorgestellt und hat den Vorteil, dass sie mit herkömmlichen medizinischen Linearbeschleunigern durchführbar ist, welche eigens dafür konfiguriert sind.\cite{nicht_so_shady_author}
Es ist eine Weiterentwicklung der IMRT, welche eine kürzere Bestrahlungsdauer zulässt, jedoch weniger Eingriff in die Planung der Bestrahlung gestattet.
Die Konfiguration ermöglicht, drei Parameter simultan anzupassen: Die Rotationsgeschwindigkeit der Gantry, die Blendenform und die Dosisrate bzw. die Intensität. 
Die Gantry kann dabei durch zeitgleiche Anpassung der Parameter schon mit einer vollen Umdrehung die benötigte Dosis an das Targetvolumen abgeben.
Mit dem Blendensystem, dass meistens aus einem System von Multilamellenkollimatoren (MLK) besteht, kann dann das bestrahlte Volumen festgelegt werden.
Dabei ist es wichtig, gesunde Zellen so gut wie möglich von der Bestrahlungsebene abzugrenzen, da diese der ionisierenden Strahlung ebenso nicht standhalten.
Das Zielvolumen kann dabei die ganze Zeit mit unterschiedlichen Intensitäten bestrahlt werden.\cite{haha_ka}
ed Arc Therapy (VMAT), 
%Image Guided Radiation Therapy (IGRT), and Stereotactic Body Radio Therapy (SBRT).



