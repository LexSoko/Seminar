\section{Diagnostik}
%willy 
%https://www.bfs.de/DE/themen/ion/anwendung-medizin/diagnostik/roentgen/roentgen-verfahren.html
\subsection{Röntgenaufnahme}
%typischer wellenlängenbereich für abbildung (was für ein material wird zur erzeugung der röntgenstrahlung benutzt)


%grund für verschiedenen Kontrast (material spezifisch)
%vergleich alte technik neue (geschichtlich)
%arten von detektoren (film folien & oder photodioden)

%typische anwendung
%gefahren 
%strahlenbelastung

\subsection{Röntgendurchleuchtung (Fluoroskopie)}
%typischer wellenlängenbereich für abbildung (was für ein material wird zur erzeugung der röntgenstrahlung benutzt)

%geschichtliche analyse
%Anwendung & technische umsetzung
%Angiographie (kontrastmittel verabreicht)
%Arten von detektoren

%gefahren
%strahlenbelastung

\subsection{Computertomographie}
%technische Umsetzung der Bildgebung und erzeugung von Röntgenstrahlen (wellenlängenbereich)
%Detektion
%schnittbildererzeugung 
%geschichtliche analyse


%gefahren
%strahlenbelastung & Low-Dose-CT

%Andere tomographische Verfahren 
%neueste entwicklung in der medizintechnik