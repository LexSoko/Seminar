%---------------DOCUMENT SETTINGS------------------------------------
%\documentclass[headheight=30pt]{scrartcl}
\documentclass[12pt]{scrartcl}

\usepackage[utf8]{inputenc} % utf8x durch utf8 ersetzt wegen biblatex
\usepackage[T1]{fontenc}
\usepackage{amsmath,amssymb,amstext,amsfonts}
\usepackage[ngerman]{babel}
\usepackage{csquotes}
\usepackage{pdfpages} %zum importieren von pdf dateien
\usepackage{geometry}
\usepackage[headsepline]{scrlayer-scrpage}
\usepackage{lastpage}
%\usepackage[style=numeric]{biblatex} %Literaturverwaltung
\usepackage{float}
\usepackage{textcomp}
\usepackage{gensymb}
\usepackage{physics}
\usepackage{graphicx}
\usepackage[export]{adjustbox}
\usepackage{a4wide}
\usepackage{siunitx}
\usepackage{hyperref}
\usepackage{multicol}
\usepackage{makecell}
%für seminararbeit
\usepackage{doi}
\usepackage[compress]{cite}

\geometry{      %holt mehr aus einer A4 Seite raus
    a4paper,
    total={170mm,257mm},
    left=20mm,
    top=25mm,
   }
\graphicspath{ {./nudes/} }  %pfad für bilder


\hypersetup{colorlinks=false}

%\addbibresource{literatur.bib} %Literatur-Resourcen
%\setlength{\headheight}{0.0pt} % macht zwar headheight warnings aber dafür nutz latex die seitengröße besser aus.
\pagestyle{scrheadings}
\newcommand{\subf}[2]{
    {
        \begin{tabular}[c]{@{}c@{}}
            {\setlength{\extrarowheight}{100pt} #1 }\\#2
        \end{tabular}
    }
}
\newcommand{\allc}{\multicolumn{1}{c|}{-}}
\newcommand{\monofig}[4]{
    
    \begin{figure}[H]
    \centering
    \includegraphics[#1]{#2}
    \caption{
        #3
    }
    \label{#4}
    \end{figure}
    
}
\newcommand{\polyfig}[6]{
    \begin{figure}[H]
        \centering
        \begin{minipage}[b]{0.45\textwidth}
            \centering
            \includegraphics[width=0.9\textwidth]{#1}
            \caption{#2}
            \label{#3}
        \end{minipage}
        \begin{minipage}[b]{0.45\textwidth}
            \centering
            \includegraphics[width=0.9\textwidth]{#4}
            \caption{#5}
            \label{#6}
        \end{minipage}
    \end{figure}
}
\newcommand{\code}[1]{
    \texttt{#1}
}

\date{\today{}, Location}
\author{Wilhelm Ecker , Aleksey Sokolov}
\title{Dokumentitel}

%---------------HEADER TEXT------------------------------------
\clearpairofpagestyles
\ihead{\today{} \\ }
\chead{Wilhelm Ecker / Aleksey Sokolov\\Elektronen Beschleuniger in der Medizin }
\ohead{Seminar\\ }
\cfoot{\pagemark \, / \, \pageref{LastPage}}

%---------------DOCUMENT TEXT------------------------------------
\begin{document}
%\includepdf[]{Deckblatt.pdf} %Insert title page NaWi-Graz
\tableofcontents
\newpage

\label{sec:Titelblatt}

\begin{titlepage}
    \centering
    \vspace{6cm}
    {\scshape\Large Hochenergie (MeV) X-Ray Diagnostik und Therapie \par}
    \vspace{3cm}
    {\Large Wilhelm Ecker Matr.: 12004188 \par } 
    {\Large Aleksey Sokolov Matr.: 12004091 \par}
    \vspace{2cm}
    {\includegraphics[width=6cm]{TUGRAZ.JPG}\par}
    
    \vfill
    {\large Seminar: Wissenschaftliches Arbeiten und Präsentationstechnik (WS22/23) \par}
    {\large Gruppe Brossmann\par}
    {\large Physik von Teilchenbeschleunigern\par}
    {\large \today\par} 
\end{titlepage}
\section{Abstract}

Seit der Entdeckung der Röntgenstrahlung zählt diese zu den wichtigsten Methoden innerhalb technischer und medizinischer Anwendungen.
Von Strukturanalysen in der Materialwissenschaft bis hin zu hochauflösenden Bildern und Animationen in der Röntgendiagnostik haben 
sie die Wissenschaft und die Industrie stark geprägt. In dieser Arbeit wird sich mit mehreren Methoden der Röntgendiagnostik und 
Therapie in der Medizin beschäftigt. Insbesondere wird der Fokus auf den Beginn der 
Diagnostik mittels Folientechnik und dessen Entwicklung bis hin zu modernster Technik im Rahmen der Computertomographie gelegt, sowie
deren Anwendungsmöglichkeiten in der Therapie von Erkrankungen. Desweiteren wird der prinzipielle Aufbau eines herkömmlichen medizinischen Elektronenlinearbeschleuniger besprochen.
Dieser wurde im letzten Jahrhundert enorm weiterentwickelt, was neue, präzisere Behandlungsmethoden schafft.
\section{Diagnostik}
\label{sec:Diagnostik}
%willy 
%https://www.bfs.de/DE/themen/ion/anwendung-medizin/diagnostik/roentgen/roentgen-verfahren.html
Röngtenaufnahmen sind seit der Entdeckung 1895 ein fixer Bestandteil der Medizin. Vorallem Fortschritte in der Computertechnologie
ermöglichen es, immer bessere bildgebende Verfahren zu entwickeln, die die Breite der radiologischen Untersuchungen ernorm vergrößert
haben. Mit bis zu 80 \% stellen großflächige radiografische Aufgnahmen von Organen Skelett und Lunge die Hauptanteil in der radiologischen
Diagnostik dar \cite{MedizinischePhysik}.

\subsection{Röntgenaufnahme}
\label{sec:Röntgenaufnahme}
Bis heute stellt die Film- und Folientechnik eine weit genutzte Technik in der Diagnostik dar. Der Vorteil hierbei liegt hierbei
bei der einfachen technik für die Aufnahme, der hohen Bildqualität, sowie preiswerten Verhältnis der Nutzen zu den Kosten. Allerdings 
wird diese Technik in Industrieländern durch digitale Detektoren ersetzt, aufgrund der schlechen Möglichkeit für die Nachbearbeitung, 
der aufwenigen chemischen Entwicklung, sowie des hohen Dosisbedarfs \cite{MedizinischePhysik}.



\subsubsection{Film-Folientechnik}
\label{subsubsec:filmfolientechnik}
Ausgang ist hierbei ein Film oder ein Fluoreszenschirm als Empfänger für das Bild. Trifft Röngtenstrahlung auf den Film,
färbt sich dieser schwarz, je nach Dosis der Strahlung. Tifft die Strahlung auf Knochen oder Gewebe, wird diese absorbiert und 
und erscheint im Bild weiß oder gräulich. Da die Bildsignal direkt proportional zur Strahlungsdosis ist, so muss eine hohe Dosis 
aufgebracht werden, um gute Bilder zu erzeugen. Die System ist optimiert worden, indem man für den Film Verstärkerfolien und
für das Fluoreszentild den Röngtenbildverstärker entwickelt hat \cite{Medizintechnik}. 

\subsubsection{Speicherfolien}
\label{subsubsec:Speicherfolien}
Bei dieser Technologie werden anstatt von Röngtenfilmen, Speicherfolien belichtet \cite{MedizinischePhysik}. Die Strahlung hebt die Elektronen in der
Kristallstruktur in ein höheres Energieniveau, wo diese für mehrere Stunden verweilen können. Um die Information auszulesen, wird 
die Folie wird aus der Kasette entfernt und mittels eines Laserstrahls abgetastet. Bei diesem Vorgang kehren die Elektronen unter
Abgabe von Fluoreszenlicht in ihren Ausgangszustand zurück. Dieses Licht wird von Photodetektoren detektiert, welche
im nächsten Schritt dieses in elektrische Signale umwandeln und digitalisieren. Nachdem der Vorgang beendet ist, kann die Folie 
mithilfe von sichtbarem Licht in den Ausgangszustand zurückversetzt und in die Kasette zurückgegeben werden \cite{Medizintechnik}.


\subsubsection{Flachdetektoren}
\label{subsubsec:Flachdetektoren}

Bei Flachdetektoren mit direkter Wandlung kommt es aufgrund einer Selenschicht zu sofortigen Umwandlung von Röntgenquanten in
elektrische Ladung. Trifft ein Röngtenquant auf die Schicht, wird eine Ladung erzeugt, in einem Kondensator gespeichert wird. Mithilfe
eines Analog-Digital-Wandlers wird die Ladung weitergeleitet und kann zeilenweise ausgelesen werden. Somit kann ein digitales Bild
erstellt werden \cite{Artikel}. 

Bei den Detektoren mit direkter Wandlung liegt eine Szintillatorschicht aus Cäsiumjodid auf Siliziumphotodioden, die als Matrix
angeordnet sind. In der Szintillatorschicht werden die Röngtenquanten in Lichtquanten umgewandelt, welche auf Detektorelemente umgeleitet 
werden. Die Ladung wird aus den Photdioden ausgelesen und mithilfe eines Analog-Digital-Wandlers in ein digitales Bild umgewandelt
 \cite{Artikel}.


%typischer wellenlängenbereich für abbildung (was für ein material wird zur erzeugung der röntgenstrahlung benutzt)


%grund für verschiedenen Kontrast (material spezifisch)
%vergleich alte technik neue (geschichtlich)
%arten von detektoren (film folien & oder photodioden)

%typische anwendung
%gefahren 
%strahlenbelastung

\subsection{Röntgendurchleuchtung (Fluoroskopie)}
\label{sec:Fluoroskopie}
Im Unterschied zu Röngtenaufnahmen kommt es bei der Fluoroskopie zu einer kontinuierlichen Strahlungsaussetzung mit röntgenstrahlung
, welches ein durchlaufendes Bild am Monitor liefert. Um ansonsten unsichtbare Organe am Bild sichtar zu machen, so werden Kontrastmittel eingesetzt.
Iodbasierte Materialien werden für Aterien, sowie die Herzkammern eingesetzt, bariumbasierte Mittel finden Anwendung im Magen-,Darmtrakt
\cite{Artikel2}.




\subsubsection{Dynamische Festkörperdetektoren}
\label{subsubsec:Festkörperdetektoren}
Diese nutzen ebenso Cäsiumjodid als Szintillator und erlauben den Betrieb nahe bei starken Magnetfeldern, zum Beispiel
wie bei Kernspintomographen. Sie sind geeignet für die Aufnahme von 3D-Bildern, aufgrund ihrer verzerrungs und gleichmäßigen
Bilddarstellung. Angewendung finden sie in der Zahn- und HNO-Medizin, sowie bei Durchleuchtung und Angiographie 
\cite{MedizinischePhysik}.

%typischer wellenlängenbereich für abbildung (was für ein material wird zur erzeugung der röntgenstrahlung benutzt)

%geschichtliche analyse
%Anwendung & technische umsetzung
%Angiographie (kontrastmittel verabreicht)
%Arten von detektoren

%gefahren
%strahlenbelastung
\subsection{Strahlenbelastung für Röntenaufnahmen und Fluoroskopien}
\label{subsec:Strahlenbelastung}

Für die Untersuchungen mittels Röngtenaufnahmen und Fluoroskopien gelten je nach Anwendungsbereich am Körper unterschiedliche 
Richtwerte. Die Strahlungsdosis wird in der SI-Einheit 'Gray', abgekürzt 'Gy' angegeben. Für Messungen am Skelett gilt eine maximale
Empfängerdosis von $\leq$10 $\mu$Gy / Bild. Für den Rumpf und Kopf gelten $\leq$5 $\mu$Gy, sowie für die digitale Durchleuchtung $\leq$0.6 
 $\mu$Gy/s. Bei digitalen mammographischen Untersuchungen liegend die Grenzwerte bei $\leq$75 $\mu$Gy und $\leq$100 $\mu$Gy \cite{Artikel}.




\subsection{Computertomographie (CT)}
\label{sec:CT}
Die Computertomographie ist neben der klassischen Röngtendiagnostik eine der bedeutensten Methoden zur Bildgebung in der 
Diagnostik von Krankheiten \cite{Artikel3,Artikel4}. Bei dieser Technik wird der Patient mittels eines Röntgenstrahls 
abgetastet, welcher kreisförmig um ihn herumfährt. Mithilfe von optischen Effekten wird der Strahl so geformt, dass dieser flach 
aufgetragen wird und die Schichtdicke angibt. Ein Teil der Röntgenstrahlung wird beim Durchdringen des Körpers absorbiert und kann 
von den Detektoren bestimmt werden. Die an den Detektoren aufgenommene Strahlung kann im Anschluss zu Graustufen umgerechnet und 
bildlich dargestellt werden \cite{Artikel3}. 
Heutzutage können Schichtdicken von bis zu 0.24 mm gemessen werden, die gemittelte Strahlendosis beträgt hierbei 2-3 mSv am Körperstamm.
Verwendung findet die CT bei der Bestimmung von Knochenbrüchen bei Gelenken, des Beckens oder der Wirbelsäule. Ebenso findet man sie 
in der Orthopädik, zum Beispiel zur Beurteilung von postoperativen Zuständen und auch in der Tumordiagnostik \cite{Artikel4}.



%technische Umsetzung der Bildgebung und erzeugung von Röntgenstrahlen (wellenlängenbereich)
%Detektion
%schnittbildererzeugung 
%geschichtliche analyse


%gefahren
%strahlenbelastung & Low-Dose-CT

%Andere tomographische Verfahren 
%neueste entwicklung in der medizintechnik
\section{X-Ray Therapie}
%aleksey
%https://www.radiologyinfo.org/en/info/linac

%https://altairusa.com/role-of-the-linear-accelerator-linac-in-cancer-radiation-therapy/
\subsection{Allgemein}

%geschichtlicher überblick
%behandelte krankheiten
%typische energieskalen

\subsection{Röntgentherapie}
Die Röntgentherapie hat sich in dem letzten Jahrhundert als einer der wichtigsten medizinischen Instrumente etabliert.
Vorallem im Bereich der Krebsbehandlung bietet sie ein gezielte und operationfreie Methode um lokal Zellenstrukturen zu verändern.
Im Laufe einer Krebsbehandlung werden über 50\% \ref{hoskin2019external} der Patienten mit einer Form der Röntgentherapie behandelt.
Meist wird dies in Kombination mit einer präzisen Diagnostikmethode, wie der im Kapitel \ref{sec:Diagnostik} behandelten Computer Tomographie (CT), sowie
der Magnetresonanztomographie (MRT) oder Positron Emissions Tomographie (PET) vollzogen.



\subsection{Funktionsweise der verwendeten Geräte}
%LinAC grobe funktionsweise
%adaptierung in der medizin
%häufig verwendete geräte im europäischen raum

\subsection{Anwendung und Arten von Behandlungsmethoden}
%Intensity-Modulated Radiation Therapy (IMRT), 
%Volumetric Modulated Arc Therapy (VMAT), 
%Image Guided Radiation Therapy (IGRT), Stereotactic Radiosurgery (SRS) and Stereotactic Body Radio Therapy (SBRT).

\subsection{Sichersheitsrisiken}
%strahlenbelastung


\section{Zusammenfassung}
Zum Abschluss kann man sagen, dass die Diagnostik mittels Röntgenstrahlung sowie die Röntgentheraphie ein großer Meilenstein in der Medizin sind.
Vorallem die bildgebenden Verfahren der Röntgendiagnostik haben Medizinern eine Möglichkeit gegeben, zerstörungsfrei bösartiges Gewebe zu erkennen und zu studieren.
Neuartige Behandlungsmöglichkeiten in der Röntgentherapie werden immer präziser und automatisierter, was durch moderne Computeralgorithmen das Wirkvolumen immer genauer lokal eingrenzt.
Diese Entwicklungen sind womöglich neben modernen Impfstoffen die größten Lebensreter unserer Zeit. 
\bibliography{literatur.bib}
\bibliographystyle{seminarstyle}
\end{document}