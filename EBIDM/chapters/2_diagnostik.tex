\section{Diagnostik}
\label{sec:Diagnostik}
%willy 
%https://www.bfs.de/DE/themen/ion/anwendung-medizin/diagnostik/roentgen/roentgen-verfahren.html
Röngtenaufnahmen sind seit der Entdeckung 1895 ein fixer Bestandteil der Medizin. Vorallem Fortschritte in der Computertechnologie
ermöglichen es, immer bessere bildgebende Verfahren zu entwickeln, die die Breite der radiologischen Untersuchungen ernorm vergrößert
haben. Mit bis zu 80 \% stellen großflächige radiografische Aufgnahmen von Organen Skelett und Lunge die Hauptanteil in der radiologischen
Diagnostik dar \cite{Medizinische Physik}.

\subsection{Röntgenaufnahme}
\label{sec:Röntgenaufnahme}
Bis heute stellt die Film- und Folientechnik eine weit genutzte Technik in der Diagnostik dar. Der Vorteil hierbei liegt hierbei
bei der einfachen technik für die Aufnahme, der hohen Bildqualität, sowie preiswerten Verhältnis der Nutzen zu den Kosten. Allerdings 
wird diese Technik in Industrieländern durch digitale Detektoren ersetzt, aufgrund der schlechen Möglichkeit für die Nachbearbeitung, 
der aufwenigen chemischen Entwicklung, sowie des hohen Dosisbedarfs \cite{Medizinische Physik}.



\subsubsection{Film-Folientechnik}
\label{subsubsec:filmfolientechnik}
Ausgang ist hierbei ein Film oder ein Fluoreszenschirm als Empfänger für das Bild. Trifft Röngtenstrahlung auf den Film,
färbt sich dieser schwarz, je nach Dosis der Strahlung. Tifft die Strahlung auf Knochen oder Gewebe, wird diese absorbiert und 
und erscheint im Bild weiß oder gräulich. Da die Bildsignal direkt proportional zur Strahlungsdosis ist, so muss eine hohe Dosis 
aufgebracht werden, um gute Bilder zu erzeugen. Die System ist optimiert worden, indem man für den Film Verstärkerfolien und
für das Fluoreszentild den Röngtenbildverstärker entwickelt hat \cite{Medizintechnik}. 

\subsubsection{Speicherfolien}
\label{subsubsec:Speicherfolien}
Bei dieser Technologie werden anstatt von Röngtenfilmen, Speicherfolien belichtet \cite{Medizinische Physik}. Die Strahlung hebt die Elektronen in der
Kristallstruktur in ein höheres Energieniveau, wo diese für mehrere Stunden verweilen können. Um die Information auszulesen, wird 
die Folie wird aus der Kasette entfernt und mittels eines Laserstrahls abgetastet. Bei diesem Vorgang kehren die Elektronen unter
Abgabe von Fluoreszenlicht in ihren Ausgangszustand zurück. Dieses Licht wird von Photodetektoren detektiert, welche
im nächsten Schritt dieses in elektrische Signale umwandeln und digitalisieren. Nachdem der Vorgang beendet ist, kann die Folie 
mithilfe von sichtbarem Licht in den Ausgangszustand zurückversetzt und in die Kasette zurückgegeben werden \cite{Medizintechnik}.


\subsubsection{Flachdetektoren}
\label{subsubsec:Flachdetektoren}

Bei Flachdetektoren mit direkter Wandlung kommt es aufgrund einer Selenschicht zu sofortigen Umwandlung von Röntgenquanten in
elektrische Ladung. Trifft ein Röngtenquant auf die Schicht, wird eine Ladung erzeugt, in einem Kondensator gespeichert wird. Mithilfe
eines Analog-Digital-Wandlers wird die Ladung weitergeleitet und kann zeilenweise ausgelesen werden. Somit kann ein digitales Bild
erstellt werden \cite{Artikel}. 

Bei den Detektoren mit direkter Wandlung liegt eine Szintillatorschicht aus Cäsiumjodid auf Siliziumphotodioden, die als Matrix
angeordnet sind. In der Szintillatorschicht werden die Röngtenquanten in Lichtquanten umgewandelt, welche auf Detektorelemente umgeleitet 
werden. Die Ladung wird aus den Photdioden ausgelesen und mithilfe eines Analog-Digital-Wandlers in ein digitales Bild umgewandelt
 \cite{Artikel}.


%typischer wellenlängenbereich für abbildung (was für ein material wird zur erzeugung der röntgenstrahlung benutzt)


%grund für verschiedenen Kontrast (material spezifisch)
%vergleich alte technik neue (geschichtlich)
%arten von detektoren (film folien & oder photodioden)

%typische anwendung
%gefahren 
%strahlenbelastung

\subsection{Röntgendurchleuchtung (Fluoroskopie)}
\label{sec:Fluoroskopie}
\subsubsection{Dynamische Festkörperdetektoren}
\label{subsubsec:Festkörperdetektoren}


%typischer wellenlängenbereich für abbildung (was für ein material wird zur erzeugung der röntgenstrahlung benutzt)

%geschichtliche analyse
%Anwendung & technische umsetzung
%Angiographie (kontrastmittel verabreicht)
%Arten von detektoren

%gefahren
%strahlenbelastung
\subsection{Strahlenbelastung für Röntenaufnahmen und Fluoroskopien}
\label{subsec:Strahlenbelastung}

Für die Untersuchungen mittels Röngtenaufnahmen und Fluoroskopien gelten je nach Anwendungsbereich am Körper unterschiedliche 
Richtwerte. Die Strahlungsdosis wird in der SI-Einheit 'Gray', abgekürzt 'Gy' angegeben. Für Messungen am Skelett gilt eine maximale
Empfängerdosis von $\leq$10 $\mu$Gy / Bild. Für den Rumpf und Kopf gelten $\leq$5 $\mu$Gy, sowie für die digitale Durchleuchtung $\leq$0.6 
 $\mu$Gy/s. Bei digitalen mammographischen Untersuchungen liegend die Grenzwerte bei $\leq$75 $\mu$Gy und $\leq$100 $\mu$Gy




\subsection{Computertomographie}
\label{sec:CT}


%technische Umsetzung der Bildgebung und erzeugung von Röntgenstrahlen (wellenlängenbereich)
%Detektion
%schnittbildererzeugung 
%geschichtliche analyse


%gefahren
%strahlenbelastung & Low-Dose-CT

%Andere tomographische Verfahren 
%neueste entwicklung in der medizintechnik