\section{X-Ray Therapie}
%aleksey
%https://www.radiologyinfo.org/en/info/linac

%https://altairusa.com/role-of-the-linear-accelerator-linac-in-cancer-radiation-therapy/
\subsection{Allgemein}

%geschichtlicher überblick
%behandelte krankheiten
%typische energieskalen

\subsection{Röntgentherapie}
Die Röntgentherapie hat sich in dem letzten Jahrhundert als einer der wichtigsten medizinischen Instrumente etabliert.
Vorallem im Bereich der Krebsbehandlung bietet sie ein gezielte und operationfreie Methode um lokal Zellenstrukturen zu verändern.
Im Laufe einer Krebsbehandlung werden über 50\% \ref{hoskin2019external} der Patienten mit einer Form der Röntgentherapie behandelt.
Meist wird dies in Kombination mit einer präzisen Diagnostikmethode, wie der im Kapitel \ref{sec:Diagnostik} behandelten Computer Tomographie (CT), sowie
der Magnetresonanztomographie (MRT) oder Positron Emissions Tomographie (PET) vollzogen.



\subsection{Funktionsweise der verwendeten Geräte}
%LinAC grobe funktionsweise
%adaptierung in der medizin
%häufig verwendete geräte im europäischen raum

\subsection{Anwendung und Arten von Behandlungsmethoden}
%Intensity-Modulated Radiation Therapy (IMRT), 
%Volumetric Modulated Arc Therapy (VMAT), 
%Image Guided Radiation Therapy (IGRT), Stereotactic Radiosurgery (SRS) and Stereotactic Body Radio Therapy (SBRT).

\subsection{Sichersheitsrisiken}
%strahlenbelastung

